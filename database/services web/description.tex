\chapter{Description des services web}
\section{Remarques générales}
\subsection{Format des données transmises ou reçues}

\subsection{Codes d'erreur}


\section{Recherche des échantillons}
\label{sampleSearch}
\subsection{Variables de recherche}
Les variables peuvent être indiquées soit directement dans la requête GET, soit dans le champ \textit{jsonval}, encodé en base 64.

% \usepackage{array} is required
\begin{longtable}{|c|c|>{\raggedright\arraybackslash}p{8cm}|}
\hline 
Code & Type & Description \\ 
\hline \endhead
uid & integer & Identifiant unique de l'échantillon dans l'instance distante \\ 
\hline 
ident & varchar & identifiant \og métier\fg{} principal \\
\hline
guid & UUID & identifiant unique quel que soit la base de données \\
\hline
uidstart & integer & uid inférieur pour une recherche sur une fourchette d'identifiants \\
\hline
uidend & integer & uid supérieur pour une recherche sur une fourchette d'identifiants \\
\hline
datestart & yyyy-mm-dd & date de début pour une recherche sur une fourchette de dates \\
\hline
dateend & yyyy-mm-dd & date de fin pour une recherche sur une fourchette de dates \\
\hline
projectid & integer & identifiant du projet ou de la sous-collection (code issu du résultat du service web \ref{projectList}) \\
\hline
sampletypeid & integer & identifiant du type d'échantillon (code issu du résultat du service web \ref{sampleTypeList}) \\
\hline
idtype & integer & type d'identifiant (code issu du résultat du service web \ref{idtype})\\
\hline
idval & varchar & identifiant recherché selon le type spécifié dans \textit{idtype}\\
\hline
\end{longtable} 

Exemple de fichier : 
\begin{lstlisting}
{"uidstart": 25,
"uidend": 50,
"datestart": "2017-01-01",
"dateend": "2017-06-30",
"idtype": 2,
"idval": "AB01"
}
\end{lstlisting}

\subsection{Données en retour}
Collection Json avec, pour chacun, les informations suivantes :
\begin{longtable}{|c|c|>{\raggedright\arraybackslash}p{6cm}|}
\hline 
Code & Type & Description \\ 
\hline \endhead
uid & integer & Identifiant dans l'instance \\
\hline
identifier & varchar & identifiant principal \og métier\fg{} \\
\hline
guid & uuid & code d'identification global \\
\hline
ids & collection & liste de tous les identifiants secondaires, selon la forme idtype: idval \\
\hline
project & varchar & nom du projet ou de la sous-collection correspondante \\
\hline
createdate & yyyy-mm-dd hh:mm:ss & date de création de l'échantillon dans la base de données d'origine \\
\hline
collectdate & yyyy-mm-dd hh:mm:ss & date de collecte ou de génération de l'échantillon\\
\hline
DATAMETIER & objet json & données \og métier\fg{} rattachées à l'échantillon \\
\hline
sampleparent & objet json & json de même structure que ce tableau comprenant les informations du parent (imbrication des différents parents le cas échéant)\\
\hline
storageproduct & varchar & produit de stockage utilisé \\
\hline
clp & varchar & risques associés aux produit de stockage \\
\hline
subsampletype & varchar & type de sous-échantillonnage \\
\hline
subsampleunit & varchar & unité de sous-échantillonnage \\
\hline
subsampleqty & double & quantité de sous-échantillons présents initialement \\
\hline


\end{longtable}

\section{Lecture d'un échantillon}

La requête, de type GET, contient soit en quatrième valeur, l'UID de l'échantillon à lire, soit la variable \textit{jsonget=}, dont la valeur, encodée en base 64, permet de spécifier le type d'identifiant utilisé :
\begin{longtable}{|c|>{\raggedright\arraybackslash}p{6cm}|}
\hline 
Code & Description \\ 
\hline
Clé & Nom de la variable utilisée \\
\hline
valeur associée & Valeur correspondante\\
\endhead

\end{longtable}

Les valeurs utilisables dans les champs sont les suivants :
\begin{longtable}{|c|c|>{\raggedright\arraybackslash}p{6cm}|}
\hline 
id & Format attendu dans val & Description \\ 
\hline
uid & integer & Clé utilisée dans la base de données \\
\hline
guid & uuid & Identifiant unique global \\
\hline
id & varchar & Identifiant principal de l'échantillon\\
\hline
xxx & varchar & tout code d'identifiant secondaire utilisable dans la base de données distante \\
\hline \endhead

\end{longtable}

Voici un exemple de fichier Json, avant son encodage en base 64:
\begin{lstlisting}
{"uid":25}
\end{lstlisting}

Les données en retour sont celles du service de recherche d'un échantillon.

\section{Lecture d'un jeu d'échantillons}

Il s'agit d'une variante du cas précédent. Le fichier json (encodé en base 64) est organisé pour fournir un identifiant par échantillon retourné. Voici un exemple du fichier :
\begin{lstlisting}
[{"uid":15},{"id":"A1-B2-C3"}]
\end{lstlisting}
Les données en retour sont celles du service de recherche d'un échantillon.

\section{Liste des projets ou sous-collections}
\label{projectList}

La requête est de type GET.
\subsection{Données en entrée}
La requête n'accepte pas de données en entrée.

\subsection{Données en retour}
La requête retourne la liste des projets ou sous-collections autorisées pour l'utilisateur considéré, sous la forme d'une collection Json contenant les informations suivantes :

\begin{longtable}{|c|c|>{\raggedright\arraybackslash}p{6cm}|}
\hline 
Code & Type & Description \\ 
\hline
id & integer & Identifiant interne du projet\\
\hline
val & varchar & Nom du projet\\
\hline
comment & varchar & Description\\
\hline \endhead
\end{longtable}

Exemple :
\begin{lstlisting}
[
{id:1,val:projet1,comment:"Description du projet 1"},
{id:3,val:projet3,comment:"Descripton du projet 3"}
]
\end{lstlisting}



\section{Liste des types d'échantillons}
\label{sampleTypeList}

La requête est de type GET.
\subsection{Données en entrée}
La requête n'accepte pas de données en entrée.

\subsection{Données en retour}
La requête retourne la liste des types d'échantillons, sous la forme d'une collection Json contenant les informations suivantes :

\begin{longtable}{|c|c|>{\raggedright\arraybackslash}p{6cm}|}
\hline 
Code & Type & Description \\ 
\hline
id & integer & Identifiant interne du type d'échantillon\\
\hline
val & varchar & Code du type d'échantillon\\
\hline
comment & varchar & Description\\
\hline \endhead
\end{longtable}

Exemple :
\begin{lstlisting}
[
{id:1,val:projet1,comment:"Description du projet 1"},
{id:3,val:projet3,comment:"Descripton du projet 3"}
]
\end{lstlisting}

\section{Liste des types d'identifiants}
\label{idtype}

La requête est de type GET.
\subsection{Données en entrée}
La requête n'accepte pas de données en entrée.

\subsection{Données en retour}
La requête retourne la liste des types d'identifiants utilisés dans la base de données, sous la forme d'une collection Json contenant les informations suivantes :

\begin{longtable}{|c|c|>{\raggedright\arraybackslash}p{6cm}|}
\hline 
Code & Type & Description \\ 
\hline
id & integer & Identifiant interne du type d'identifiant\\
\hline
val & varchar & Code de l'identifiant\\
\hline
comment & varchar & Description\\
\hline \endhead
\end{longtable}

Exemple :
\begin{lstlisting}
[
{id:1,val:cab,comment:"Code a barres utilise pour l'inventaire"},
{id:2,val:igsn,comment:"Code IGSN pour les carottes de prelevement"}
]
\end{lstlisting}

