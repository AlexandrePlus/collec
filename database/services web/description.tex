\chapter{Description des services web}

\section{Remarques générales}

Les services web sont destinés à fournir des informations aux clients. Leur implémentation dépend des logiciels, ainsi que la manière de traiter les données.

Pour des raisons de compatibilité entre les différents logiciels, si un argument fourni figure dans la requête mais n'est pas utilisable par le service web, ce dernier ne doit pas rejeter la requête, sauf si aucun argument ne permet de discriminer la recherche demandée.

\subsection{Format des données transmises ou reçues}
Les données sont échangées au format JSON.

Les requêtes pour obtenir des informations sont de type GET. Actuellement, aucun service web n'est prévu pour mettre à jour des informations dans la base distante.

\subsection{Codes d'erreur}

Voici la liste minimale des codes d'erreur génériques que les services web devraient retourner :
\begin{longtable}{|c|>{\raggedright\arraybackslash}p{10cm}|}
\hline 
Code &  Description \\ 
\hline \endhead
400 & Bad argument : un des arguments fournis n'est pas valide, ou aucun argument utilisable n'a été indiqué\footnote{Dans le cas où des arguments sont fournis, mais non utilisables par le service web considéré, l'exécution de la requête conduisant à renvoyer trop d'informations} \\
\hline
401 & Unauthorized : les droits d'accès sont insuffisants pour permettre d'obtenir les informations \\
\hline
404 & Not found : la ressource demandée est introuvable\\
\hline
408 & RequestTimeOut : la requête n'a pas abouti dans le temps imparti, ou elle a été annulée par le client\\
\hline
500 & InternalError : une erreur d'exécution ne permet pas de retourner le résultat de la requête\\
\hline
503 & ServiceUnavailable : le service est temporairement indisponible\\
\hline
\end{longtable} 

\section{Table des listes techniques}

Ce service web permet de récupérer la liste des tables de référence qui peuvent être utilisées pour lancer une recherche.

\subsection{url type}
\url{https://sitedistant.fr/sw/v1/params}

\subsection{Données en entrée}
Le service web ne nécessite pas de paramètre d'entrée.


\subsection{Données en retour}
La requête retourne la liste demandée sous la forme d'une collection Json contenant les informations suivantes :

\begin{longtable}{|c|c|>{\raggedright\arraybackslash}p{6cm}|}
\hline 
Code & Type & Description \\ 
\hline
code & varchar & Code de la table de référence. C'est ce code qui sera utilisé pour obtenir la liste des valeurs disponibles dans le service web de récupération du contenu\\
\hline
val & varchar & Texte affichable dans un masque de recherche\\
\hline
valfr & varchar & Texte affichable dans un masque de recherche, en français\\
\hline
valen & varchar & Texte affichable dans un masque de recherche, en anglais\\
\hline \endhead
\end{longtable}

Exemple :
\begin{lstlisting}
[
{code:"project",val:"Projets", valen:"projects"},
{code:"sampletype",val:"Types d'echantillons", valen:"samples types"},
{code:"identifiertype",val:"Identifiants metier", valen:"business identifiers"},
{code:"place",val:"Lieux de prelevement des echantillons",valen:"sampling places"}
]
\end{lstlisting}

\section{Listes techniques}
\label{listes}

Ce service web permet de récupérer une liste de référence nécessaire pour comprendre ou rechercher les échantillons de la base distante.

\subsection{url type}
\url{https://sitedistant.fr/sw/v1/reference/project} ou 
\url{https://sitedistant.fr/sw/v1/reference/var=project}.

La valeur fournie (\textit{project} ici) est une des valeurs récupérée dans le service web précédent (rubrique \textit{code}).

La requête est de type GET. 


\subsection{Données en retour}
La requête retourne la liste demandée sous la forme d'une collection Json contenant les informations suivantes :

\begin{longtable}{|c|c|>{\raggedright\arraybackslash}p{10cm}|}
\hline 
Code & Type & Description \\ 
\hline
id & integer & Identifiant interne\\
\hline
val & varchar & Libellé\\
\hline
comment & varchar & Description ou commentaire\\
\hline \endhead
\end{longtable}

Exemple :
\begin{lstlisting}
[
{id:1,val:"projet1",comment:"Description du projet 1"},
{id:3,val:"projet3",comment:"Descripton du projet 3"}
]
\end{lstlisting}


\section{Rechercher des échantillons}
\label{sampleSearch}
\subsection{url type}
\url{https://sitedistant.fr/sw/v1/sample_search}

\subsection{Variables de recherche}
Les variables sont fournies soit dans la requête GET, soit sous la forme d'une variable JSON nommée \textit{val}.

% \usepackage{array} is required
\begin{longtable}{|c|c|>{\raggedright\arraybackslash}p{8cm}|}
\hline 
Code & Type & Description \\ 
\hline \endhead
uid & integer & Identifiant unique de l'échantillon dans l'instance distante \\ 
\hline 
ident & varchar & identifiant \og métier\fg{} principal \\
\hline
guid & UUID & identifiant unique quel que soit la base de données \\
\hline
uidstart & integer & uid inférieur pour une recherche sur une fourchette d'identifiants \\
\hline
uidend & integer & uid supérieur pour une recherche sur une fourchette d'identifiants \\
\hline
datestart & yyyy-mm-dd & date de début pour une recherche sur une fourchette de dates \\
\hline
dateend & yyyy-mm-dd & date de fin pour une recherche sur une fourchette de dates \\
\hline
codefamily & objet Json & objet json permettant de rechercher une valeur sur une table de paramètres. \textit{Codefamily} doit être remplacé par le code retourné lors de la recherche des listes de paramètres. \\
& & L'objet JSON accepte, comme informations associées : \\
& & \textit{id} : identifiant spécifique à rechercher (par exemple, le code de l'identifiant métier). \textit{id} peut être facultatif, si la recherche ne porte que sur la valeur, par exemple, le lieu de prélèvement; \\
& & \textit{val} : valeur recherchée\\
& & si uniquement \textit{val} est indiqué, on peut utiliser un couple classique: \textit{codefamily:val}\\
\hline
\end{longtable} 

Exemple de fichier : 
\begin{lstlisting}
{"uidstart": 25,
"uidend": 50,
"datestart": "2017-01-01",
"dateend": "2017-06-30",
"identifiertype":{id:10, val:"AB01"},
"sampletype":2
}
\end{lstlisting}

La recherche portera sur les échantillons dont l'UID est entre 25 et 50, dont la date est entre le 1\ier{} janvier et le 30 juin 2017, dont l'identifiant métier connu sous le numéro 10 vaut AB01, et dont le type d'échantillons est 2.

\subsection{Données en retour}
Collection Json avec, pour chacun, les informations suivantes :
\begin{longtable}{|c|c|>{\raggedright\arraybackslash}p{6cm}|}
\hline 
Code & Type & Description \\ 
\hline \endhead
uid & integer & Identifiant dans l'instance \\
\hline
identifier & varchar & identifiant principal \og métier\fg{} \\
\hline
guid & uuid & code d'identification global \\
\hline
ids & collection & liste de tous les identifiants secondaires, selon la forme idtype: idval \\
\hline
project & varchar & nom du projet ou de la sous-collection correspondante \\
\hline
createdate & yyyy-mm-dd hh:mm:ss & date de création de l'échantillon dans la base de données d'origine \\
\hline
collectdate & yyyy-mm-dd hh:mm:ss & date de collecte ou de génération de l'échantillon\\
\hline
metadata & objet json & données \og métier\fg{} rattachées à l'échantillon \\
\hline
sampleparent & objet json & json de même structure que ce tableau comprenant les informations du parent (imbrication des différents parents le cas échéant)\\
\hline
storageproduct & varchar & produit de stockage utilisé \\
\hline
clp & varchar & risques associés aux produit de stockage \\
\hline
subsampletype & varchar & type de sous-échantillonnage \\
\hline
subsampleunit & varchar & unité de sous-échantillonnage \\
\hline
subsampleqty & double & quantité de sous-échantillons présents initialement \\
\hline
samplingplacename & varchar & nom du lieu de collecte \\
\hline
\end{longtable}

\section{Lire un échantillon}

\subsection{url type}
\url{https://sitedistant.fr/sw/v1/sample}
\subsection{Variables de recherche}
La requête, de type GET, contient soit en dernière valeur l'UID de l'échantillon à lire, soit une variable GET dont le nom correspond à l'identifiant recherché et la valeur associée
Les valeurs utilisables dans les champs sont les suivants :
\begin{longtable}{|c|c|>{\raggedright\arraybackslash}p{6cm}|}
\hline 
id & Format attendu dans val & Description \\ 
\hline
uid & integer & Clé utilisée dans la base de données \\
\hline
guid & uuid & Identifiant unique global \\
\hline
id & varchar & Identifiant principal de l'échantillon\\
\hline
xxx & varchar & tout identifiant secondaire disponible\\
\hline \endhead

\end{longtable}

Voici un exemple de fichier Json:
\begin{lstlisting}
{"uid":25}
\end{lstlisting}
\subsection{Données en retour}
Les données en retour sont celles du service de recherche d'un échantillon.

\section{Lire un jeu d'échantillons}
\subsection{url type}
\url{https://sitedistant.fr/sw/v1/samples?param=[{xxx}]}
\subsection{Variables de recherche}
Il s'agit d'une variante du cas précédent. Un fichier JSON est fourni dans la variable \textit{param}, organisé pour fournir un identifiant par échantillon retourné. Voici un exemple du fichier :
\begin{lstlisting}
[{"uid":15},{"id":"A1-B2-C3"}]
\end{lstlisting}
\subsection{Données en retour}
Les données en retour sont celles du service de recherche d'un échantillon.



